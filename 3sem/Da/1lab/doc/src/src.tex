\section{Описание}

Идея реализации сортировки подсчетом такова: создается дополнительный массив, в котором мы будем записывать колличество каждых элементов, в нашем случае $short\,int$ ключей. Так как мы знаем границы этих чисел, проблем это не составит. Дальше в цикле пробегаем массив ключей и инкрементируем соотвествующее значение в дополнительном массиве. Следующий цикл, добавляем к каждой ячейке $i$ значение ячейки $i - 1$. Проверка правильности выполнения алгоритма: число в последней ячейке является колличестом элементов первоначального массива. Для правильного свапа, в начале функции сортировки обьявляем два дополнительных массива типа $short\,int$ и $char\,*$ размерами с первоначальные. В них мы будем записывать результаты свапа, после чего перепишем их в первоначальные поданные на сортировку массивы. Как итог за 4 цикла мы получаем два отсортированных вектора, которые мы потом переписываем в изначально поданные.

\pagebreak

\section{Исходный код}

Для решения задачи необходимо было написать два класса, вектор для $short\,int$ и для $char\,*$. Их функционал: создание(выделение динамической памяти под структуру с обьявлением локальных переменных), добавление элемента в массив, расширение массива, удаление массива(удаление всех его динамических элементов и освобождения указателя и присвоения ему $NULL$).

Далее следует сама функция сортировки описанная в методе решения. Принимает на вход векторы ключей и значений. Если посмотреть код который я отправил на чекер то там я старался обьяснить каждую сторчку кода, потому, что я, в отличие от остальных, решил писать на чистом Си, что ввиду использования  стрелок $(->)$ при работе с указателями вызывает у читающего код легкий ступор от происходящего.

\begin{lstlisting}[language=C]
#include <stdlib.h>
#include <stdio.h>
#include <string.h>

typedef struct STRING { 
    char ** array;
    int size;
    int vars;
} STRING;

STRING * create_str (int s) { 
    STRING * string = (STRING *)malloc(sizeof(STRING));
    string->array = (char **)malloc(sizeof(char *) * s); 
    for (int i = 0; i < s; i++)
        string->array[i] = (char*)malloc(sizeof(char) * 2049); 
    string->size = s; 
    string->vars = 0;
    return string;
}

STRING * resize_str (STRING * string) { 
    if (string->size == 0) {  string->size++; } 
    else  {  string->size *= 2;  } 
    string->array = (char **)realloc(string->array, sizeof(char *) * string->size); 
    return string;
}

STRING * push_str (STRING * string, char * value) { 
    if (string->size == 0 || string->vars + 1 > string->size)
        string = resize_str(string); 
    string->array[string->vars] = (char *)malloc(sizeof(char) * 2049); 
    strcpy(string->array[string->vars], value); 
    string->vars++; 
    return string;
}

void delete_str (STRING ** string) { 
    for (int i = 0; i < (*string)->vars; i++)
        free((*string)->array[i]); 
    free(((*string)->array)); 
    free((*string)); 
    (*string) = NULL;
}

typedef struct SHINT {
    int * array;
    int size;
    int vars;
} SHINT;

SHINT * create_arr (int s) {
    SHINT * arr = (SHINT *)malloc(sizeof(SHINT));
    arr->array = (int *)malloc(sizeof(int) * s);
    arr->size = s;
    arr->vars = 0;
    return arr;
}

SHINT * resize_arr (SHINT * arr) { 
    if (arr->size == 0) {  arr->size++; } 
    else  {  arr->size *= 2;  } 
    arr->array = (int *)realloc(arr->array, sizeof(int) * arr->size); 
    return arr;
}

SHINT * push_arr (SHINT * arr, int value) { 
    if (arr->size == 0 || arr->vars + 1 > arr->size)
        arr = resize_arr(arr); 
    arr->array[arr->vars] = value; 
    arr->vars++; 
    return arr;
}

void delete_arr (SHINT ** arr) { 
    free(((*arr)->array)); 
    (*arr)->array = NULL; 
    free(*arr); 
    *arr = NULL; 
}

void stableString(SHINT * v, STRING * s) {
    int max = 65536;
    SHINT * c = create_arr(0); 
    SHINT * b = create_arr(0); 
    STRING * bb = create_str(0); 
    for (int i = 0; i < max; i++)
        push_arr(c, 0); 
    for (int i = 0; i < v->vars; i++) {
        c->array[v->array[i]] += 1; 
        push_arr(b, 0); 
        push_str(bb, ""); 
    }
    for (int i = 1; i < max; i++)
        c->array[i] = c->array[i] + c->array[i - 1]; 
    for (int i = v->vars - 1; i > -1; i--) {
        c->array[v->array[i]] = c->array[v->array[i]] - 1;
        b->array[c->array[v->array[i]]] = v->array[i];
        strcpy(bb->array[c->array[v->array[i]]], s->array[i]); 
    }
    for (int i = 0; i < v->vars; i++) { 
        v->array[i] = b->array[i];
        strcpy(s->array[i], bb->array[i]);
    }
    delete_arr(&c);
    delete_arr(&b);
    delete_str(&bb);
}

int main(void) {
    SHINT * arr = create_arr(0); 
    STRING * string = create_str(0); 
    int d; 
    char * s = (char *)malloc(sizeof(char) * 2049); 
    while (scanf("%d\t%s", &d, s) == 2) { 
        if (d >= 0 && d <= 65535) { 
            push_arr(arr, d);
            push_str(string, s);
        }
    }
    free(s); 
    stableString(arr, string); 
    for (int i = 0; i < arr->vars; i++) 
        printf("%d\t%s\n", arr->array[i], string->array[i]);
    delete_arr(&arr); 
    delete_str(&string); 
    return 0;
}
\end{lstlisting}

\pagebreak

\section{Консоль}
\begin{alltt}
MacBook-Pro-Pavel:1lab pavelgamov\$ gcc -pedantic -Wall -std=c99 \newline -Werror -Wno-sign-compare -lm main.c -o prog
MacBook-Pro-Pavel:1lab pavelgamov\$ cat test.txt 
999 hellopapapa_adfadf
0   hellllooasowewecsadf
4	11f2erbiuebri123uerbi2ebiberibier
1	aaaa
65535	vls
1234	qerqefd

65535	anodtherone
808 bassszzz
777 space
1233    space_here_dude
3256432  wefsgdhnsearf

-10 fsdbfgds
MacBook-Pro-Pavel:1lab pavelgamov\$ ./prog < test.txt 
0	hellllooasowewecsadf
1	aaaa
4	11f2erbiuebri123uerbi2ebiberibier
777	space
808	bassszzz
999	hellopapapa_adfadf
1233	space_here_dude
1234	qerqefd
65535	vls
65535	anodtherone
\end{alltt}

% 1-5 Попытки: ошибка компиляции ввиду неэффективного и неправильного кода в классах векторов $short int$ и $char *$.

% 6 - 10 Попытки: дошел до прошел тест 4, оказывается верхняя планка по размеру значения 2048 чаров берется включительно, увеличил до 2049 и дошел до 11 теста где чекер показывал неправильный ответ.

% 11 - 16 Попытки: изменил $short int$ на $int$ и ввел проверку на границы входных данных вручную. Теперь если инт не входит в данное значение от $0$ до $65535$ включительно, то элементы просто не добавляются. Также заменил цикл $while$, при считывании теперь он будет проверят кол-во полученных значений и если оно не равно двум, что включает в себя также условия конца файла $(EOF)$, то считывание заканчивается и программа начинает сортировать полученные векторы. С другой стороны это не обрабатывает введение пустых строк, но от подачи неполных данных спасает. Программа зашла на чекер.%


\pagebreak