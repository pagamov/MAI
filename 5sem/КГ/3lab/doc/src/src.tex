\section{Решение}

Фигура строится в два этапа, отрисовка боковых сторон, а также двух оснований, рисуем полигонами по кругу, из трегольников составляя основания, находящиеся на расстоянии H друг от друга, также по окружностям радиуса R.


Ползунком n регулируется точность и мелкость шага отрисовки.

\includegraphics[scale=0.5]{pictures/1.png}

Приведена фугура с малой точностью отрисовки.

\includegraphics[scale=0.4]{pictures/2.png}

Приведена фугура с высокой точностью отрисовки.

% \section{Исходный код}

% \begin{lstlisting}[language=C++]
% \end{lstlisting}

% \lstset{language=[gnu] make}
% \lstset{
%   language=[gnu] make,
%   keywordstyle=\color{teal}\textbf,
%   stringstyle=\color{blue},
%   identifierstyle=\itshape
% }

% \begin{lstlisting}
% CC = g++
% CCFLAGS = -std=c++14 -Wall -pedantic -O3
% ###____###
% solution : main.cpp *.hpp ; $(CC) $(CCFLAGS) main.cpp -o solution
% clean	 : ;
% \end{lstlisting}

% \section{Консоль}

% \begin{alltt}
% \end{alltt}

% \pagebreak