\CWHeader{Лабораторная работа \textnumero 9}

\CWProblem{

Вариант 7 \\ Поиск максимального потока алгоритмом Форда-Фалкерсона

Разработать программу на языке C или C++, реализующую указанный алгоритм согласно заданию:

Задан взвешенный ориентированный граф, состоящий из $n$ вершин и $m$ ребер. Вершины пронумерованы целыми числами от 1 до $n$. Необходимо найти величину максимального потока в графе при помощи алгоритма Форда-Фалкерсона. Для достижения приемлемой производительности в алгоритме рекомендуется использовать поиск в ширину, а не в глубину. Истоком является вершина с номером 1, стоком – вершина с номером $n$. Вес ребра равен его пропускной способности. Граф не содержит петель и кратных ребер.

Формат входных данных

В первой строке заданы 1 $\leq$ $n$ $\leq$ 2000 и 1 $\leq$ $m$ $\leq$ 10000. В следующих $m$ строках записаны ребра. Каждая строка содержит три числа – номера вершин, соединенных ребром, и вес данного ребра. Вес ребра – целое число от 0 до $10^{9}$.

Формат результата

Необходимо вывести одно число – искомую величину максимального потока. Если пути из истока в сток не существует, данная величина равна нулю.

Примеры \\
Входные данные \\
5 6 \\ 1 2 4 \\ 1 3 3 \\ 1 4 1 \\ 2 5 3 \\ 3 5 3 \\ 4 5 10 \\
Результат работы \\ 7
}

\pagebreak