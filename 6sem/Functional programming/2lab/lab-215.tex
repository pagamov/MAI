%%% пример отчета взян с сайта http://k806.ystok.ru/

\documentclass[15pt]{extarticle}

\usepackage{fullpage}
\usepackage{multicol,multirow}
\usepackage{tabularx}
\usepackage{ulem}
\usepackage[utf8]{inputenc}
\usepackage[russian]{babel}
\usepackage{amsmath}
\usepackage{amssymb}

% margin from paper borders
\usepackage[margin={0.8in, 0.3in}]{geometry}
% displaystyle в каждом мас моде
\everymath{\displaystyle}

\usepackage{titlesec}

\titleformat{\section}
  {\normalfont\Large\bfseries}{\thesection.}{0.3em}{}

\titleformat{\subsection}
  {\normalfont\large\bfseries}{\thesubsection.}{0.3em}{}

\titlespacing{\section}{0pt}{*2}{*2}
\titlespacing{\subsection}{0pt}{*1}{*1}
\titlespacing{\subsubsection}{0pt}{*0}{*0}
\usepackage{listings}
\lstloadlanguages{Lisp}
\lstset{extendedchars=false,
    breaklines=true,
    breakatwhitespace=true,
    keepspaces = true,
    tabsize=2
}



\begin{document}

\section*{Отчет по лабораторной работе №\,2
по курсу \guillemotleft  Функциональное программирование\guillemotright}

\begin{flushright}
Студент группы 8О-307 МАИ \textit{Гамов Павел}, \textnumero 4 по списку \\
\makebox[7cm]{Контакты: {\tt pagamov@gmail.com} \hfill} \\
\makebox[7cm]{Работа выполнена: 25.03.2021 \hfill} \\
\ \\
Преподаватель: Иванов Дмитрий Анатольевич, доц. каф. 806 \\
\makebox[7cm]{Отчет сдан: \hfill} \\
\makebox[7cm]{Итоговая оценка: \hfill} \\
\makebox[7cm]{Подпись преподавателя: \hfill} \\

\end{flushright}

\section{Тема работы}
Простейшие функции работы со списками Коммон Лисп

\section{Цель работы}
Научиться конструировать списки, находить элемент в списке, использовать схему линейной и древовидной рекурсии для обхода и реконструкции плоских списков и деревьев.

\section{Задание (вариант № 2.15/2)}
Запрограммируйте рекурсивно на языке Коммон Лисп функцию, подсчитывающую число вхождений заданного действительного числа в дерево. Действительное означает, что типы чисел могут смешиваться: целые, рациональные и с плавающей точкой.

(count-real 3.0 '((1 2.2) 3.0 (4 5/6 (3 6)) 7)) => 2

\section{Оборудование студента}
macOS Catalina 10.15.7 Intel Core i5 2.3 GHz 8 ГБ RAM

\section{Программное обеспечение}
macOS, среда {\tt vim + sbcl}

\section{Идея, метод, алгоритм}
Надо понять как правильно разбирать вложенные списки, есть встроенные методы car и cdr, позволяющие взять начало и конец списка, также есть способ проверки списка на нулевой список equal '(), проверка на равенство числа выполняется с помощью предиката equalp, который игнорирует разные типы чисел.

\section{Сценарий выполнения работы}
Для начала я расписал условия для разных входных данных. Может прийти пустой список, в данном случае функция возвращает ноль. Если список не пустой, отсекаем начало и конец и в зависимости от того, чем является начало списка, число или список, возвращаем сумму результатов обработки внутренних списков и чисел.

\section{Распечатка программы и её результаты}

\subsection{Исходный код}

\begin{lstlisting}
(defun count-real (num l)
    (if (equal l '())
      0
      (if (listp (car l))
        (+ (count-real num (car l)) (count-real num (cdr l)))
        (if (equalp (car l) num)
          (+ 1 (count-real num (cdr l)))
          (count-real num (cdr l))))))

(print (count-real 3.0 '(1 2 3.0))) ; => 1
(print (count-real 3.0 '())) ; => 0
(print (count-real 3.0 '((1) 2 9/3))) ; => 1
(print (count-real 3.0 '(1 2 (3)))) ; => 1
(print (count-real 3.0 '((1 2.2) 3.0 (4 5/6 (3 6)) 7))) ; => 2
\end{lstlisting}

\subsection{Результаты работы}

\begin{lstlisting}
1 
0 
1 
1 
2
\end{lstlisting}

\section{Дневник отладки}

\begin{tabular}{|c|c|c|c|}
\hline
Дата     & Событие              & Действие по исправлению   & Примечание \\
30.04.21 & Написал рекурсивную    & Разобрал все случаи &            \\
         & обработку дерева &                           &            \\
\hline
\end{tabular}

\section{Замечания автора по существу работы}
Работа со списками в Lisp очень похожа синтаксис языка Prolog, отсекание головы и хвоста (H,T)

\section{Выводы}
Внутренние функции обработки списков в языке Lisp, позволяют организовать эффективную работу с ними.

\end{document}
