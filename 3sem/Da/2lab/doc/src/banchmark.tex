\section{Тест производительности}

Для уже написанной программы допишем второй файл на языке с++, где заменим наше дерево встроенным в с++ map. Также слегка модифицируем оба файла так чтобы на вход первым параметром он принимал кол-во поступаемых элементов и потом используя \#include <time.h> напишем простой таймер от начала работы до ее завершения. Программа написанная на питоне генерирует нужное вол-во тестов и прогоняет их в программах, которые выводят по завершению работы время и кол-во элементов. 

\begin{alltt}
MacBook-Pro-Pavel:src pavelgamov make test
python tester.py

How many loop?	5
Start size?	10000

gcc    -c -o code/main.o code/main.c
gcc    -c -o code/m.o code/m.c
gcc    -c -o code/delete.o code/delete.c
gcc    -c -o code/insert.o code/insert.c
gcc    -c -o code/clear.o code/clear.c
gcc    -c -o code/saveload.o code/saveload.c
gcc    -c -o code/merge.o code/merge.c
gcc -Wall -pedantic -std=c99 -g code/*.o -o main

MacBook-Pro-Pavel:src pavelgamov cat res.txt 
size:  10000		time:  0.009841
size:  20000		time:  0.019643
size:  40000		time:  0.044685
size:  80000		time:  0.086463
size:  160000		time:  0.168590

MacBook-Pro-Pavel:src pavelgamov cat res2.txt
size:  10000		time:  0.262778
size:  20000		time:  0.553230
size:  40000		time:  1.037304
size:  80000		time:  2.074010
size:  160000		time:  4.231900
\end{alltt}

То ли я напортачил при создании прогарммы с использованием $map$, либо моя программа работает быстрее и эффективнее.

\pagebreak
