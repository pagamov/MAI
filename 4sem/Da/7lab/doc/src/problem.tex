\CWHeader{Лабораторная работа \textnumero 7}

\CWProblem{

При помощи метода динамического программирования разработать алгоритм решения задачи, определяемой своим вариантом; оценить время выполнения алгоритма и объем затрачиваемой оперативной памяти. Перед выполнением задания необходимо обосновать применимость метода динамического программирования.
\newline

Разработать программу на языке C или C++, реализующую построенный алгоритм. Формат входных и выходных данных описан в варианте задания:
\newline

У вас есть рюкзак, вместимостью $m$, а так же $n$ предметов, у каждого из которых есть вес $w_{i}$ и стоимость $c_{i}$. Необходимо выбрать такое подмножество $I$ из них, чтобы: $ \sum_{i \in I} w_{i} \leq m $. $(\sum_{i \in I} c_{i}) * |I|$ является максимальной из всех возможных. $|I|$ – мощность множества $I$.
\newline

Формат входных данных
В первой строке заданы $1 \leq n \leq 100$ и $1 \leq m \leq 5000$. В последующих n строках через пробел заданы параметры предеметов: $w_{i}$ и $c_{i}$.
\newline

Формат результата
В первой строке необходимо вывести одно число – максимальное значение $(\sum_{i \in I} c_{i}) * |I|$, а на второй – индексы предметов, входящих в ответ.

Примеры
\newline

Входные данные

3 6

2 1

5 4

4 2

Результат работы

6

1 3

}

\pagebreak