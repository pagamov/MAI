%%% пример отчета взян с сайта http://k806.ystok.ru/

\documentclass[15pt]{extarticle}

\usepackage{fullpage}
\usepackage{multicol,multirow}
\usepackage{tabularx}
\usepackage{ulem}
\usepackage[utf8]{inputenc}
\usepackage[russian]{babel}
\usepackage{amsmath}
\usepackage{amssymb}
\usepackage{graphicx}

% margin from paper borders
\usepackage[margin={0.8in, 0.3in}]{geometry}
% displaystyle в каждом мас моде
\everymath{\displaystyle}

\usepackage{titlesec}

\titleformat{\section}
  {\normalfont\Large\bfseries}{\thesection.}{0.3em}{}

\titleformat{\subsection}
  {\normalfont\large\bfseries}{\thesubsection.}{0.3em}{}

\titlespacing{\section}{0pt}{*2}{*2}
\titlespacing{\subsection}{0pt}{*1}{*1}
\titlespacing{\subsubsection}{0pt}{*0}{*0}
\usepackage{listings}
\lstloadlanguages{Lisp}
\lstset{extendedchars=false,
    breaklines=true,
    breakatwhitespace=true,
    keepspaces = true,
    tabsize=2
}



\begin{document}

\section*{Отчет по лабораторной работе №\,4
по курсу \guillemotleft  Функциональное программирование\guillemotright}

\begin{flushright}
Студент группы 8О-307 МАИ \textit{Гамов Павел}, \textnumero 4 по списку \\
\makebox[7cm]{Контакты: {\tt pagamov@gmail.com} \hfill} \\
\makebox[7cm]{Работа выполнена: 30.05.2021 \hfill} \\
\ \\
Преподаватель: Иванов Дмитрий Анатольевич, доц. каф. 806 \\
\makebox[7cm]{Отчет сдан: \hfill} \\
\makebox[7cm]{Итоговая оценка: \hfill} \\
\makebox[7cm]{Подпись преподавателя: \hfill} \\

\end{flushright}

\section{Тема работы}
Знаки и строки

\section{Цель работы}
Научиться работать с литерами (знаками) и строками при помощи функций обработки строк и общих функций работы с последовательностями.

\section{Задание (вариант № 4.15/2)}
Запрограммировать на языке Коммон Лисп функцию-предикат, принимающую один аргумент - текст.
Функция должна возвращать истину, если текст содержит литеры, отличные от латинских и русских букв и пробела.


\section{Оборудование студента}
macOS Catalina 10.15.7 Intel Core i5 2.3 GHz 8 ГБ RAM

\section{Программное обеспечение}
macOS, среда {\tt vim + sbcl}

\section{Идея, метод, алгоритм}
Создаем переменную found которая будет считать количество символов которые не входят в алфавиты. Так как мы работаем с текстом, делаем первый dolist разбивая текст на предложения, далее для каждого слова, которые мы разбиваем функцией word-list, которую я взял с сайта, мы пробегаем по каждому символу и если он не входит в алфавиты, прибавляем единицу к found. Далее пройдя все чары, если переменная found больше нуля, возвращается True иначе False.

\section{Сценарий выполнения работы}
Разобрался с тем как правильно итерировать по словам, далее включил функцию word-list, на самом деле ее можно было избежать, можно итерировать прямо по строке, вместо слов.

\section{Распечатка программы и её результаты}

\subsection{Исходный код}

\begin{lstlisting}
(defun whitespace-char-p (char)
  (member char '(#\Space #\Tab #\Newline)))

(defun word-list (string)
  (loop with len = (length string)
        for left = 0 then (1+ right)
        for right = (or (position-if #'whitespace-char-p string
                                     :start left)
                        len)
        unless (= right left)	; исключить пустые слова
          collect (subseq string left right)
        while (< right len)))

(defun is-ordinar (char)
    (or (member char (coerce "ABCDEFGHIGKLMNOPQRSTUVWXYZ" 'list))
        (member char (coerce "abcdefghigklmnopqrstuvwxyz" 'list))
        (member char (coerce "АБВГДЕЁЖЗИЙКЛМНОПРСТУФХЦЧШЩЪЫЬЭЮЯ" 'list))
        (member char (coerce "абвгдеёжзийклмнопрстуфхцчшщъыьэюя" 'list))))

(defun if-text-is-ordinary(txt)
    (let ((found 0))
        (dolist (sentence txt)
            (dolist (word (word-list sentence))
                (loop for c across word do
                    (if (not (is-ordinar c))
                        (setf found (+ found 1))))))
        (> found 0)))

(print (if-text-is-ordinary '("русский текст, где есть знаки препинания" "тут тоже есть, ведь так?")))
(print (if-text-is-ordinary '("русский текст без знаков препинания" "тут тоже их нет")))

(print (if-text-is-ordinary '("eng text with punctuation marks." "here another one!")))
(print (if-text-is-ordinary '("eng text without punctuation marks." "тут тоже их нет")))
\end{lstlisting}

\subsection{Результаты работы}

\begin{lstlisting}
T
NIL
T
NIL
\end{lstlisting}

\section{Дневник отладки}

\begin{tabular}{|c|c|c|c|}
\hline
Дата     & Событие    & Действие по исправлению   & Примечание \\
\hline
& & & \\
\hline
\end{tabular}

\section{Замечания автора по существу работы}
Функции доступа к строке похожи на Си-шный вариант. Работать удобно, много встроенных функций изменения, добавления, разбора строки, а также функций сравнения и включения символов.

\section{Выводы}
В этой лабораторной я научился работать со строками на языке Lisp. Данная работа показала эффективность работы языка с типом строки.

\end{document}
