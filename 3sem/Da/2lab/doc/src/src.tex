\section{Описание}

Идея реализации - создание главного файла main.c, где мы будем считывать строки (команды) и числа и отправлять их через анализатор в структуру, допустим мы читаем +, следовательно дальше мы читаем строку в буфер и число, которые будут добавлены в дерево. Отдельно поговорим про саму структуру дерева, я разбил ее на темаические файлы, в которых реализован функционал структуры. 

Файл clear.c содержит в себе функции удаления памяти в структуре, очистки нодов и рекурсивные очистки и удаления поддеревьев. 

delete.c содежит в себе функции удаления элементов из дерева, включая в себя как легкие - удаление из листа, так и удаление из не листьев, требующих перестройки и перекомпановки дерева, посредством слияний и перебросок из левых или правых деревьев.

insert.c содержит функционал по добавлению элементов в структуру, добавление в лист, сильное добавление требуется для добавления в нод элемента с фиксированными хвостами и ссылками на потомков (требуется в некоторых случаях при переброске элементов при слиянии нодов), функция разделения дерева на элемент и два его потомка (требуется после вставки для устранения переполнения при добавлении элемента, таким образом если добавив элемент окажется, что нод переполнен, вызывается функция деления нода).

m.c - общий файл который нужен для описания более общих функций которые нужны для преобразования входных данных, функции сравнения строк между собой, функции бинарного поиска, который требуется для более быстрого нахождения строк в нодах, при больших T это играет большую роль, так как операция сравнения занимает слишком много времени. Также главная функция обьявления и выделения памяти под структуру, а также функция проверки принадлежности элемента к дереву.

merge.c - содержит в себе лишь одну функцию по слиянию двух нодов в один, содержит под собой 6 случаев, включая нулевые.

saveload.c - функционал для осуществления бинарной серриализации и бинарной десерриализации структуры из/в файл. Для сохранения дерева в файл и дальнейшей 



\pagebreak

\section{Исходный код}

Далее представлен код программы в соответствии с разбиением по файлам для удобства навигации.

\begin{lstlisting}[language=C]
//bTree.h
#ifndef BTREE_H
#define BTREE_H

#include <stdio.h>
#include <stdlib.h>
#include <string.h>

typedef struct T_Node {
    char * key;
    unsigned long long value;
} T_Node;

typedef struct T_B_Tree {
    int t_size;
    int list;
    struct T_B_Tree * parent;
    int capacity;
    struct T_Node ** object;
    struct T_B_Tree ** ptr;
} T_B_Tree;


// clear.c
void DeletebTree(T_B_Tree ** tree);
void DeleteSubbTree(T_B_Tree * tree);
void Clear(T_B_Tree * tree);

// delete.c
void Merge(T_B_Tree * tree);
void DeleteFromBTree(T_B_Tree * tree, char * key);
T_B_Tree * FindNode(T_B_Tree * tree, char * key);
void Erase(T_B_Tree * tree, char * key);
T_B_Tree * FindLeft(T_B_Tree * tree, T_B_Tree * tar);
T_B_Tree * FindRight(T_B_Tree * tree, T_B_Tree * tar);
int FindDiv(T_B_Tree * parent, T_B_Tree * tar1, T_B_Tree * tar2);
void EasyCase(T_B_Tree * tree, char * key);
void BadCase(T_B_Tree * tree, char * key);

// insert.c
int InsertBTree(T_B_Tree * tree, char * key, unsigned long long value, T_B_Tree * left, T_B_Tree * right);
void InsertBTreeforce(T_B_Tree * tree, char * key, unsigned long long value, T_B_Tree * left, T_B_Tree * right);
int DivideBTree(T_B_Tree * tree);
int AddBTree(T_B_Tree * tree, char * key, unsigned long long value, T_B_Tree * left, T_B_Tree * right);

// m.c
char Toolover(char a);
void RegDown(char * str);
int StrcmpReg(char * str1, char * str2);
int BinarySearch(T_B_Tree * tree, char * key);
T_B_Tree * CreateBTree(int t_s);
int InBTree(T_B_Tree * tree, char * key);
T_Node * FindSpot(T_B_Tree * tree, char * key);
T_B_Tree * RestrictBTree(T_B_Tree * tree);
void Swap(T_B_Tree * tree, int a, int b);
void Swup(T_B_Tree * tree, int a, int b);

// merge.c
void Merge(T_B_Tree * tree);

// saveload.c
void Save(T_B_Tree * tree, char * file);
void SaveSub(T_B_Tree * tree, FILE * f);
T_B_Tree * Load(char * file);
void LoadSub(T_B_Tree * tree, FILE * f, int size, int list);

#endif
\end{lstlisting}

\begin{lstlisting}[language=C]
//clear.c
#include "bTree.h"

void Clear(T_B_Tree * tree) {
    for (int i = 0; i < 2 * tree->t_size; i++) {
        free(tree->object[i]->key);
        free(tree->object[i]);
    }
    free((tree)->object);
    free((tree)->ptr);
}

void DeleteSubbTree(T_B_Tree * tree) {
    if (tree != NULL) {
        for (int i = 0; i < tree->capacity + 1; i++) {
            DeleteSubbTree(tree->ptr[i]);
            tree->ptr[i] = NULL;
        }
        Clear(tree);
        free(tree);
    }
}

void DeletebTree(T_B_Tree ** tree) {
    if ((*tree) != NULL) {
        for (int i = 0; i < (*tree)->capacity + 1; i++) {
            DeleteSubbTree((*tree)->ptr[i]);
            (*tree)->ptr[i] = NULL;
        }
        Clear(*tree);
        free(*tree);
        *tree = NULL;
    }
}
\end{lstlisting}

\begin{lstlisting}[language=C]
//delete.c
#include "bTree.h"

void DeleteFromBTree(T_B_Tree * tree, char * key) {
    T_B_Tree * ptr = FindNode(tree, key);
    if (ptr->list) {
        if (ptr->capacity < tree->t_size - 1 && ptr->capacity > 0) {
            Erase(ptr, key);
            return;
        }
        if (ptr->capacity > tree->t_size - 1) {
            Erase(ptr, key);
        } else if (ptr->capacity == tree->t_size - 1) {
            EasyCase(ptr, key);
        }
    } else {
        BadCase(ptr, key);
    }
}

T_B_Tree * FindNode(T_B_Tree * tree, char * key) {
    int pos = BinarySearch(tree, key);
    if (pos < 0) {
        return tree;
    } else {
        return FindNode(tree->ptr[pos], key);
    }
}

void Erase(T_B_Tree * tree, char * key) {
    int pos = BinarySearch(tree, key);
    int i = -(pos + 1);
    for (int j = i ; j < tree->capacity - 1; j++) {
        Swup(tree, j, j + 1);
    }
    tree->capacity--;
}

T_B_Tree * FindLeft(T_B_Tree * tree, T_B_Tree * tar) {
    int i = 0;
    while (i < tree->capacity + 1) {
        if (tree->ptr[i] == tar) {
            return i != 0 ? tree->ptr[i - 1] : NULL;
        }
        i++;
    }
    return NULL;
}

T_B_Tree * FindRight(T_B_Tree * tree, T_B_Tree * tar) {
    int i = 0;
    while (i < tree->capacity + 1) {
        if (tree->ptr[i] == tar) {
            return i != tree->capacity ? tree->ptr[i + 1] : NULL;
        }
        i++;
    }
    return NULL;
}

int FindDiv(T_B_Tree * parent, T_B_Tree * tar1, T_B_Tree * tar2) {
    for (int i = 0; i < parent->capacity; i++) {
        if (parent->ptr[i] == tar1 && parent->ptr[i + 1] == tar2) {
            return i;
        }
    }
    return -1;
}

void EasyCase(T_B_Tree * tree, char * key) {
    if (tree->parent == NULL) {
        Erase(tree, key);
    } else {
        T_B_Tree * l = FindLeft(tree->parent, tree);
        T_B_Tree * r = FindRight(tree->parent, tree);
        int l_div = FindDiv(tree->parent, l, tree);
        int r_div = FindDiv(tree->parent, tree, r);
        if (r != NULL) {
            if (r->capacity > tree->t_size - 1) {
                Erase(tree, key);
                InsertBTree(tree, tree->parent->object[r_div]->key, tree->parent->object[r_div]->value, NULL, NULL);
                strcpy(tree->parent->object[r_div]->key, r->object[0]->key);
                tree->parent->object[r_div]->value = r->object[0]->value;
                Erase(r, tree->parent->object[r_div]->key);
                return;
            }
        }
        if (l != NULL) {
            if (l->capacity > tree->t_size - 1) {
                Erase(tree, key);
                InsertBTree(tree, tree->parent->object[l_div]->key, tree->parent->object[l_div]->value, NULL, NULL);
                strcpy(tree->parent->object[l_div]->key, l->object[l->capacity - 1]->key);
                tree->parent->object[l_div]->value = l->object[l->capacity - 1]->value;
                Erase(l, l->object[l->capacity - 1]->key);
                return;
            }
        }
        if (r != NULL) {
            if (r->capacity == tree->t_size - 1) {
                if (tree->parent->capacity <= tree->t_size - 1) {
                    Merge(tree->parent);
                }
                r = FindRight(tree->parent, tree);
                r_div = FindDiv(tree->parent, tree, r);
                Erase(tree, key);
                InsertBTree(tree, tree->parent->object[r_div]->key, tree->parent->object[r_div]->value, NULL, NULL);
                for (int i = 0; i < r->capacity; i++) {
                    InsertBTree(tree, r->object[i]->key, r->object[i]->value, r->ptr[i], r->ptr[i + 1]);
                }
                r->capacity = 0;
                Erase(tree->parent, tree->parent->object[r_div]->key);
                tree->parent->ptr[r_div] = tree;
                Clear(r);
                free(r);
                return;
            }
        }
        if (l != NULL) {
            if (l->capacity == tree->t_size - 1) {
                if (tree->parent->capacity <= tree->t_size - 1) {
                    Merge(tree->parent);
                }
                l = FindLeft(tree->parent, tree);
                l_div = FindDiv(tree->parent, l, tree);
                Erase(tree, key);
                InsertBTree(l, tree->parent->object[l_div]->key, tree->parent->object[l_div]->value, NULL, NULL);
                for (int i = 0; i < tree->capacity; i++)
                    InsertBTree(l, tree->object[i]->key, tree->object[i]->value, tree->ptr[i], tree->ptr[i + 1]);
                tree->capacity = 0;
                for (int i = 0; i < l->capacity; i++)
                    InsertBTreeforce(tree, l->object[i]->key, l->object[i]->value, l->ptr[i], l->ptr[i + 1]);
                l->capacity = 0;
                Erase(tree->parent, tree->parent->object[l_div]->key);
                tree->parent->ptr[l_div] = tree;
                Clear(l);
                free(l);
                return;
            }
        }
    }
    return;
}

void BadCase(T_B_Tree * tree, char * key) {
    T_B_Tree * pivl = NULL;
    T_B_Tree * pivr = NULL;

    int pos = BinarySearch(tree, key);
    int i = -(pos + 1);

    if (tree->ptr[i]->capacity > tree->t_size - 1) {
        pivl = tree->ptr[i];
        pivr = tree->ptr[i + 1];
        InsertBTree(pivr, tree->object[i]->key, tree->object[i]->value, pivl->ptr[pivl->capacity], NULL);
        if (pivl->ptr[pivl->capacity] != NULL) {
            pivl->ptr[pivl->capacity]->parent = pivr;
        }
        strcpy(tree->object[i]->key, pivl->object[pivl->capacity - 1]->key);
        tree->object[i]->value = pivl->object[pivl->capacity - 1]->value;
        Erase(pivl, tree->object[i]->key);
        DeleteFromBTree(pivr, pivr->object[0]->key);
        if (pivr->capacity == 2 * tree->t_size - 1) {
            DivideBTree(pivr);
        }
        return;
    }

    if (tree->ptr[i + 1]->capacity > tree->t_size - 1) {
        pivl = tree->ptr[i];
        pivr = tree->ptr[i + 1];
        InsertBTree(pivl, tree->object[i]->key, tree->object[i]->value, NULL, pivr->ptr[0]);
        if (pivr->ptr[0] != NULL) {
            pivr->ptr[0]->parent = pivl;
        }
        strcpy(tree->object[i]->key, pivr->object[0]->key);
        tree->object[i]->value = pivr->object[0]->value;
        Erase(pivr, tree->object[i]->key);
        DeleteFromBTree(pivl, pivl->object[pivl->capacity - 1]->key);
        if (pivl->capacity == 2 * tree->t_size - 1) {
            DivideBTree(pivl);
        }
        return;
    }

    if (tree->ptr[i]->capacity == tree->t_size - 1 && tree->ptr[i + 1]->capacity == tree->t_size - 1) {
        if (tree->capacity <= tree->t_size - 1) {
            Merge(tree);
        }
        int pos2 = BinarySearch(tree, key);
        i = -(pos2 + 1);
        pivl = tree->ptr[i];
        pivr = tree->ptr[i + 1];
        InsertBTree(pivl, tree->object[i]->key, tree->object[i]->value, NULL, NULL);
        for (int j = 0; j < pivr->capacity + 1; j++) {
            if (pivr->ptr[i] != NULL) {
                pivr->ptr[i]->parent = pivl;
            }
        }
        for (int j = 0; j < pivr->capacity; j++) {
            InsertBTreeforce(pivl, pivr->object[j]->key, pivr->object[j]->value, pivr->ptr[j], pivr->ptr[j + 1]);
        }
        Clear(pivr);
        free(pivr);
        Erase(tree, tree->object[i]->key);
        tree->ptr[i] = pivl;
        DeleteFromBTree(pivl, pivl->object[tree->t_size - 1]->key);
        if (pivl->capacity == 2 * tree->t_size - 1) {
            DivideBTree(pivl);
        }
        return;
    }
    return;
}
\end{lstlisting}

\begin{lstlisting}[language=C]
//insert.c
#include "bTree.h"

int InsertBTree(T_B_Tree * tree, char * key, unsigned long long value, T_B_Tree * left, T_B_Tree * right) {
    int i = 0;
    if (tree->capacity == 0) {
        strcpy(tree->object[i]->key, key);
        tree->object[i]->value = value;
        tree->ptr[i] = left;
        tree->ptr[i + 1] = right;
        tree->capacity++;
        return i;
    }
    for (i = 0; i < tree->capacity; i++) {
        if (StrcmpReg(key, tree->object[i]->key) == 1) {
            for (int j = tree->capacity; j != i; j--)
                Swap(tree, j - 1, j);
            strcpy(tree->object[i]->key, key);
            tree->object[i]->value = value;
            tree->ptr[i] = left;
            tree->capacity++;
            return i;
        }
    }
    strcpy(tree->object[i]->key, key);
    tree->object[i]->value = value;
    tree->ptr[i + 1] = right;
    tree->capacity++;
    return i;
}

void InsertBTreeforce(T_B_Tree * tree, char * key, unsigned long long value, T_B_Tree * left, T_B_Tree * right) {
    int pos = InsertBTree(tree, key, value, left, right);
    tree->ptr[pos] = left;
    tree->ptr[pos + 1] = right;
}

int DivideBTree(T_B_Tree * tree) {
    T_B_Tree * nleft = CreateBTree(tree->t_size);
    if (tree->list == 0) {
        nleft->list = 0;
    }
    for (int i = tree->t_size; i < 2 * tree->t_size - 1; i++) {
        InsertBTree(nleft, tree->object[i]->key, tree->object[i]->value, tree->ptr[i], tree->ptr[i + 1]);
        if (tree->ptr[i] != NULL) {
            tree->ptr[i]->parent = nleft;
        }
        if (tree->ptr[i + 1] != NULL) {
            tree->ptr[i + 1]->parent = nleft;
        }
        tree->capacity--;
    }
    if (tree->parent == NULL) {
        T_B_Tree * nroot = CreateBTree(tree->t_size);
        nroot->list = 0;
        tree->parent = nroot;
    }
    nleft->parent = tree->parent;
    InsertBTreeforce(tree->parent, tree->object[tree->t_size - 1]->key, tree->object[tree->t_size - 1]->value, tree, nleft);
    tree->capacity--;
    if (tree->parent->capacity == 2 * tree->t_size - 1) {
        DivideBTree(tree->parent);
    }
    return 1;
}

int AddBTree(T_B_Tree * tree, char * key, unsigned long long value, T_B_Tree * left, T_B_Tree * right) {
    if (tree != NULL) {
        if (tree->list) {
            InsertBTree(tree, key, value, left, right);
            if (tree->capacity == 2 * tree->t_size - 1) {
                DivideBTree(tree);
            }
            return 1;
        } else {
            int pos = BinarySearch(tree, key);
            return AddBTree(tree->ptr[pos], key, value, left, right);
        }
    }
    return 0;
}
\end{lstlisting}

\begin{lstlisting}[language=C]
//m.c
#include "bTree.h"

char Toolover(char a) {
    return a >= 'A' && a <= 'Z' ? a + ('a' - 'A') : a;
}

void RegDown(char * str) {
    if (str == NULL) {
        return;
    }
    int i = 0;
    while (str[i] != '\0') {
        str[i] = Toolover(str[i]);
        i++;
    }
}

int StrcmpReg(char * str1, char * str2) {
    int a = strcmp(str1, str2);
    if (a > 0) {
        return 1;
    } else if (a < 0) {
        return -1;
    }
    return 0;
}

int BinarySearch(T_B_Tree * tree, char * key) {
    int l = 0;
    int r = tree->capacity - 1;
    if (tree->capacity == 0) {
        return 0;
    }
    if (tree->capacity == 1) {
        int pos = StrcmpReg(tree->object[l]->key, key);
        if (pos == 1) {
            return l + 1;
        } else if (pos == -1) {
            return l;
        } else {
            return -l - 1;
        }
    }
    int mid;
    int res;
    while (l != r - 1) {
        mid = (l + r) / 2;
        res = StrcmpReg(tree->object[mid]->key, key);
        if (res == 1) {
            l = mid;
        } else if (res == -1) {
            r = mid;
        } else if (res == 0) {
            return -mid - 1;
        }
    }
    int lRes = StrcmpReg(tree->object[l]->key, key);
    if (lRes == 0) {
        return -l - 1;
    } else if (lRes == -1) {
        return l;
    }
    int rRes = StrcmpReg(tree->object[r]->key, key);
    if (rRes == 0) {
        return -r - 1;
    } else if (rRes == 1) {
        return r + 1;
    }
    return r;
}

T_B_Tree * CreateBTree(int t_s) {
    T_B_Tree * tree = (T_B_Tree *)calloc(1, sizeof(T_B_Tree));
    tree->t_size = t_s;
    tree->parent = NULL;
    tree->capacity = 0;
    tree->list = 1;
    tree->object = (T_Node **)calloc((2 * tree->t_size), sizeof(T_Node *));
    tree->ptr = (T_B_Tree **)calloc((2 * tree->t_size + 1), sizeof(T_B_Tree *));
    int i;
    for (i = 0; i < (2 * tree->t_size); i++) {
        tree->object[i] = (T_Node *)calloc(1, sizeof(T_Node));
        tree->object[i]->key = (char *)calloc(257, sizeof(char));
        tree->ptr[i] = NULL;
    }
    tree->ptr[i] = NULL;
    return tree;
}

int InBTree(T_B_Tree * tree, char * key) {
    if (tree != NULL) {
        int pos = BinarySearch(tree, key);
        if (pos < 0) {
            return 1;
        } else {
            return InBTree(tree->ptr[pos], key);
        }
    } else {
        return 0;
    }
}

T_Node * FindSpot(T_B_Tree * tree, char * key) {
    if (tree == NULL) {
        return NULL;
    }
    int pos = BinarySearch(tree, key);
    if (pos < 0) {
        return tree->object[-(pos + 1)];
    } else {
        return FindSpot(tree->ptr[pos], key);
    }
}

T_B_Tree * RestrictBTree(T_B_Tree * tree) {
    if (tree->capacity == 0 && tree->ptr[0] != NULL) {
        tree = tree->ptr[0];
        Clear(tree->parent);
        free(tree->parent);
        tree->parent = NULL;
        return RestrictBTree(tree);
    }
    return tree->parent != NULL ? RestrictBTree(tree->parent) : tree;
}

void Swap(T_B_Tree * tree, int a, int b) {
    char * pivc = NULL;
    unsigned long long pivn;
    pivc = tree->object[a]->key;
    pivn = tree->object[a]->value;
    tree->object[a]->key = tree->object[b]->key;
    tree->object[a]->value = tree->object[b]->value;
    tree->object[b]->key = pivc;
    tree->object[b]->value = pivn;
    tree->ptr[b + 1] = tree->ptr[b];
    tree->ptr[a + 1] = tree->ptr[a];
    return;
}

void Swup(T_B_Tree * tree, int a, int b) {
    char * pivc = NULL;
    unsigned long long pivn;
    pivc = tree->object[a]->key;
    pivn = tree->object[a]->value;
    tree->object[a]->key = tree->object[b]->key;
    tree->object[a]->value = tree->object[b]->value;
    tree->object[b]->key = pivc;
    tree->object[b]->value = pivn;
    tree->ptr[a] = tree->ptr[a + 1];
    tree->ptr[b] = tree->ptr[b + 1];
    return;
}
\end{lstlisting}

\begin{lstlisting}[language=C]
//merge.c
#include "bTree.h"

void Merge(T_B_Tree * tree) {
    if (tree == NULL) {
        return;
    }
    if (tree->parent == NULL) {
        return;
    }
    T_B_Tree * l = FindLeft(tree->parent, tree);
    T_B_Tree * r = FindRight(tree->parent, tree);
    int l_div = FindDiv(tree->parent, l, tree);
    int r_div = FindDiv(tree->parent, tree, r);

    if (r != NULL) {
        if (r->capacity > tree->t_size - 1) {
            InsertBTree(tree, tree->parent->object[r_div]->key, tree->parent->object[r_div]->value, NULL, r->ptr[0]);
            r->ptr[0]->parent = tree;
            strcpy(tree->parent->object[r_div]->key, r->object[0]->key);
            tree->parent->object[r_div]->value = r->object[0]->value;
            Erase(r, tree->parent->object[r_div]->key);
            return;
        }
    }

    if (l != NULL) {
        if (l->capacity > tree->t_size - 1) {
            InsertBTree(tree, tree->parent->object[l_div]->key, tree->parent->object[l_div]->value, l->ptr[l->capacity], NULL);
            l->ptr[l->capacity]->parent = tree;
            strcpy(tree->parent->object[l_div]->key, l->object[l->capacity - 1]->key);
            tree->parent->object[l_div]->value = l->object[l->capacity - 1]->value;
            Erase(l, tree->parent->object[l_div]->key);
            return;
        }
    }

    if (r != NULL) {
        if (r->capacity == tree->t_size - 1) {
            if (tree->parent->capacity <= tree->t_size - 1) {
                Merge(tree->parent);
            }
            r = FindRight(tree->parent, tree);
            r_div = FindDiv(tree->parent, tree, r);
            InsertBTree(tree, tree->parent->object[r_div]->key, tree->parent->object[r_div]->value, NULL, NULL);
            for (int i = 0; i < r->capacity + 1; i++) {
                r->ptr[i]->parent = tree;
            }
            for (int i = 0; i < r->capacity; i++) {
                InsertBTreeforce(tree, r->object[i]->key, r->object[i]->value, r->ptr[i], r->ptr[i + 1]);
            }
            Erase(tree->parent, tree->parent->object[r_div]->key);
            tree->parent->ptr[r_div] = tree;
            Clear(r);
            free(r);
            return;
        }
    }

    if (l != NULL) {
        if (l->capacity == tree->t_size - 1) {
            if (tree->parent->capacity <= tree->t_size - 1) {
                Merge(tree->parent);
            }
            l = FindLeft(tree->parent, tree);
            l_div = FindDiv(tree->parent, l, tree);
            InsertBTree(tree, tree->parent->object[l_div]->key, tree->parent->object[l_div]->value, NULL, NULL);
            for (int i = 0; i < l->capacity + 1; i++) {
                l->ptr[i]->parent = tree;
            }
            for (int i = 0; i < l->capacity; i++) {
                InsertBTreeforce(tree, l->object[i]->key, l->object[i]->value, l->ptr[i], l->ptr[i + 1]);
            }
            Erase(tree->parent, tree->parent->object[l_div]->key);
            tree->parent->ptr[l_div] = tree;
            Clear(l);
            free(l);
            return;
        }
    }
}
\end{lstlisting}

\begin{lstlisting}[language=C]
//saveload.c
#include "bTree.h"

void Save(T_B_Tree * tree, char * file) {
    if (tree != NULL) {
        FILE * f = fopen(file, "wb");
        if (f == NULL) {
            return;
        }
        fwrite(&(tree->t_size), sizeof(tree->t_size), 1, f);
        SaveSub(tree, f);
        fclose(f);
    }
}

void SaveSub(T_B_Tree * tree, FILE * filedis) {
    fwrite(&(tree->capacity), sizeof(tree->capacity), 1, filedis);
    fwrite(&(tree->list), sizeof(tree->list), 1, filedis);
    for (int i = 0; i < tree->capacity; i++) {
        fwrite(tree->object[i]->key, 257, 1, filedis);
        fwrite(&(tree->object[i]->value), sizeof(tree->object[i]->value), 1, filedis);
    }
    if (tree->list == 0) {
        for (int i = 0; i < tree->capacity + 1; i++) {
            SaveSub(tree->ptr[i], filedis);
        }
    }
}

T_B_Tree * Load(char * file) {
    FILE * f = fopen(file, "rb");
    if (f == NULL) {
        return NULL;
    }
    int t_s;
    fread(&(t_s), sizeof(t_s), 1, f);
    T_B_Tree * tree = CreateBTree(t_s);
    int size, list;
    fread(&(size), sizeof(size), 1, f);
    fread(&(list), sizeof(list), 1, f);
    LoadSub(tree, f, size, list);
    fclose(f);
    return tree;
}

void LoadSub(T_B_Tree * tree, FILE * f, int size, int list) {
    tree->capacity = size;
    tree->list = list;
    for (int i = 0; i < size; i++) {
        fread(tree->object[i]->key, 257, 1, f);
        fread(&(tree->object[i]->value), sizeof(tree->object[i]->value), 1, f);
    }
    if (list == 0) {
        int size_, list_;
        for (int i = 0; i < size + 1; i++) {
            fread(&(size_), sizeof(size_), 1, f);
            fread(&(list_), sizeof(list_), 1, f);
            tree->ptr[i] = CreateBTree(tree->t_size);
            tree->ptr[i]->parent = tree;
            LoadSub(tree->ptr[i], f, size_, list_);
        }
    }
}
\end{lstlisting}

\begin{lstlisting}[language=C]
//main.c
#include <stdlib.h>
#include <stdio.h>
#include "bTree.h"

int main(int argc, char ** argv) {
    int glob_size = 30;
    char * input = (char *)calloc(257, sizeof(char));
    unsigned long long number;
    T_B_Tree * tree = CreateBTree(glob_size);
    T_B_Tree * ret = NULL;
    while (scanf("%s", input) > 0) {
        if (strcmp(input, "+\0") == 0) {
            scanf("%s\t%llu", input, &number);
            RegDown(input);
            switch (InBTree(tree, input)) {
                case 1:
                    printf("Exist\n");
                    break;
                case 0:
                    switch (AddBTree(tree, input, number, NULL, NULL)) {
                        case 1:
                            tree = RestrictBTree(tree);
                            printf("OK\n");
                            break;
                        case 0:
                            printf("ERROR: in adding\n");
                            exit(1);
                            break;
                    }
                    break;
            }
        } else if (strcmp(input, "-\0") == 0) {
            scanf("%s", input);
            RegDown(input);
            if (InBTree(tree, input)) {
                DeleteFromBTree(tree, input);
                tree = RestrictBTree(tree);
                printf("OK\n");
            } else {
                printf("NoSuchWord\n");
            }
        } else if (strcmp(input, "!\0") == 0) {
            scanf("%s", input);
            if (strcmp(input, "Save\0") == 0) {
                scanf("%s", input);
                Save(tree, input);
                printf("OK\n");
            } else if (strcmp(input, "Load\0") == 0) {
                scanf("%s", input);
                ret = Load(input);
                if (ret != NULL) {
                    DeletebTree(&tree);
                    tree = NULL;
                    tree = ret;
                    ret = NULL;
                    printf("OK\n");
                } else {
                    printf("ERROR: in Load\n"); //
                }
            }
        } else {
            RegDown(input);
            T_Node * pivet = FindSpot(tree, input);
            if (pivet == NULL) {
                printf("NoSuchWord\n");
            } else {
                printf("OK:\t%llu\n", pivet->value);
            }
        }
    }
    DeletebTree(&tree);
    free(input);
    return 0;
}
\end{lstlisting}

Также представлен Makefile для компиляции всего проекта. Можно понять что все исходники находятся в папке code.

\begin{alltt}
CC = gcc
CCFLAGS = -Wall -pedantic -std=c99 -g
###____###
main		: code/main.o code/m.o code/delete.o code/insert.o code/clear.o code/saveload.o code/merge.o code/bTree.h ; $(CC) $(CCFLAGS) code/*.o -o main
m.o			: code/m.c code/bTree.h 		; $(CC) $(CCFLAGS) -c code/m.c 			; mv m.o src
delete.o	: code/delete.c code/bTree.h 	; $(CC) $(CCFLAGS) -c code/delete.c  	; mv delete.o src
insert.o	: code/insert.c code/bTree.h 	; $(CC) $(CCFLAGS) -c code/insert.c  	; mv insert.o src
clear.o 	: code/clear.c code/bTree.h		; $(CC) $(CCFLAGS) -c code/clear.c  	; mv clear.o src
main.o		: code/main.c code/bTree.h		; $(CC) $(CCFLAGS) -c code/main.c 		; mv main.o src
saveload.o	: code/saveload.c code/bTree.h	; $(CC) $(CCFLAGS) -c code/saveload.c 	; mv saveload.o src
merge.o		: code/merge.c code/bTree.h 	; $(CC) $(CCFLAGS) -c code/merge.c 		; mv merge.o src
###___###
\end{alltt}

\pagebreak

\section{Консоль}
\begin{alltt}
root@pavel:/media/sf_Coding/3_sem/Da/2lab/src# make
gcc    -c -o code/main.o code/main.c
gcc    -c -o code/m.o code/m.c
gcc    -c -o code/delete.o code/delete.c
gcc    -c -o code/insert.o code/insert.c
gcc    -c -o code/clear.o code/clear.c
gcc    -c -o code/saveload.o code/saveload.c
gcc    -c -o code/merge.o code/merge.c
gcc -Wall -pedantic -std=c99 -g code/*.o -o main

root@pavel:/media/sf_Coding/3_sem/Da/2lab/src# make gen
python generator.py
how many iterations?	10000

root@pavel:/media/sf_Coding/3_sem/Da/2lab/src# make valgrind
valgrind -s --leak-check=full ./main < text.txt > res.txt
==3637== Memcheck, a memory error detector
==3637== Copyright (C) 2002-2017, and GNU GPL'd, by Julian Seward et al.
==3637== Using Valgrind-3.15.0 and LibVEX; rerun with -h for copyright info
==3637== Command: ./main
==3637==
==3637==
==3637== HEAP SUMMARY:
==3637==     in use at exit: 0 bytes in 0 blocks
==3637==   total heap usage: 387,870 allocs, 387,870 frees, 63,376,749 bytes allocated
==3637==
==3637== All heap blocks were freed -- no leaks are possible
==3637==
==3637== ERROR SUMMARY: 0 errors from 0 contexts (suppressed: 0 from 0)
\end{alltt}

\pagebreak