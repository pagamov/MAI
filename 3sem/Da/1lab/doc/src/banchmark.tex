\section{Тест производительности}

Алгоритм использует 4 цикла. Первый заполняет нулями размер $d$, что является верхней границей допустимых чисел. Обратимся к терминологии большого $O()$, получаем временую оценку $O(d)$ Дальше следует цикл подсчета кол-ва элементов в векторе, где мы бежим от нуля до $n$, что является размером вектора ключей. Текущая оценка времени $O(n + d)$. Цикл прибавляющий к $i$ значение $i - 1$ бежит от 1 до $d$. Оценка $O(n + d + d - 1)$. В предпоследнем цикле бежим от $n$ до нуля и заполняем дополнительные векторы. После чего цикл с заполнением старых вектором из новых. Оценка $O(3n + 2d - 1)$. Не видим степеней и при оговорке что $d$ много меньше $n$ объявляем сортировку линейной за $O(n)$. По крайней мере так работает в теории. На практике наблюдается небольшой рост из за перезаписи строк в дополнительные массивы. Можно решить поставив еще один массив целых чисел, в котором будем менять местами элементы олицетворяя свап строк, просто изменится слегка функция вывода результата, получится так что вектор строк останется тем же, просто будем выводить результат по другому правилу, на основе элементов доп массива.


Для наглядной проверки на линейность была написана программа на языке $Python$, она генерирует и сосставляет тесты, которые отправляются в измененые программы, которые теперь умеют засекать время сортировки. В первой программе линейная сортировка подсчетом, во второй, быстрая сортировка Хоара.
Результаты работы программ яляется вывод в консоль кол-ва элементов и время работы, которые в итоге перенаправляются в файл $res.txt$.

\begin{alltt}
MacBook-Pro-Pavel:1lab pavelgamov python generator.py 
How many loop?	5
Start size?	10000
Here is result of sorting by main sort (left) and additional sort (right)
main sort: size: 10000	time:  0.028300		10000	time:  0.348022
main sort: size: 20000	time:  0.049244		20000	time:  0.740333
main sort: size: 40000	time:  0.099339		40000	time:  1.617800
main sort: size: 80000	time:  0.220949		80000	time:  3.468129
main sort: size: 160000	time:  0.370362		160000	time:  7.456386
MacBook-Pro-Pavel:1lab pavelgamov
\end{alltt}

Как видим Хоар справляется хуже при равных тестах.

\begin{alltt}

\end{alltt}

\pagebreak
