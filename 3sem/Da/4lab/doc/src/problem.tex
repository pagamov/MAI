\CWHeader{Лабораторная работа \textnumero 4}

\CWProblem{

Вариант №6-1

Вариант алгоритма: Поиск одного образца-маски: в образце может встречаться «джокер» (представляется символом ? — знак вопроса), равный любому другому символу. При реализации следует разбить образец на несколько, не содержащих «джокеров», найти все вхождения при помощи алгоритма Ахо-Корасик и проверить их относительное месторасположение.

Вариант алфавита: Слова не более 16 знаков латинского алфавита (регистронезависимые). Запрещается реализовывать алгоритмы на алфавитах меньшей размерности, чем указано в задании.

Формат входных данных: Искомый образец задаётся на первой строке входного файла. В случае, если в задании требуется найти несколько образцов, они задаются по одному на строку вплоть до пустой строки. Затем следует текст, состоящий из слов или чисел, в котором нужно найти заданные образцы. Никаких ограничений на длину строк, равно как и на количество слов или чисел в них, не накладывается.

Формат результата: В выходной файл нужно вывести информацию о всех вхождениях искомых образцов в обрабатываемый текст: по одному вхождению на строку. Для заданий, в которых требуется найти только один образец, следует вывести два числа через запятую: номер строки и номер слова в строке, с которого начинается найденный образец. В заданиях с большим количеством образцов, на каждое вхождение нужно вывести три числа через запятую: номер строки; номер слова в строке, с которого начинается найденный образец; порядковый номер образца. Нумерация начинается с единицы. Номер строки в тексте должен отсчитываться от его реального начала (то есть, без учёта строк, занятых образцами). Порядок следования вхождений образцов несущественен.

Примеры

Входные данные

cat ? dog

cat

cat

dog

dog

Результат

1, 1

2, 1
}

\pagebreak