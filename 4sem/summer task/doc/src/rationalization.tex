\begin{center}
\bfseries{\large МАТЕРИАЛЫ ПО РАЦИОНАЛИЗАТОРСКИМ ПРЕДЛОЖЕНИЯМ}
\end{center}

Конечно, самым главным предложением может являться введение login-pass системы, основанной на временных токенах: данная система позволит ввести контроль пользователей, а также ограничить кол-во запросов, что может дать дополнительный слой безопасности.

Введение защиты от атак на сервер: используя либо текущие алгоритмы отсеивания запросов, либо основываясь на данных, написать самому.

Введение новых middleware обработчиков: клиент не гарантирует, что читал мануал api сервера, надо уметь защитить сервер от плохих запросов, а также добавить слой фидбека клиенту, с подробной детализацией ошибки, и где он ошибся.

Увеличение производительности по поиску плагиата:

\quad 1. Используя асинхронные функции, имеем возможность параллельно искать плагиат по многим текстам.

\quad 2. Заранее преобразовывать текст в нужные форматы, пока что он хранится одной строчкой, в плане шинглирования, мы могли бы заранее делать из него матрицу слов, что ускорило бы поиск.

\quad 3. Технически можно попробывать вставить код на С++ в JS проект, еще не понятно как можно связать два этих языка, но теоретически, это может ускорить поиск в разы.

\quad 4. Усовершенствование алгоритма поиска плагиата: текущий алгоритм недостачно хорошо работает, потому что не учитывает замену букв похожими, не учитывает лексические конструкции, не вырезает цитаты и сноски, не работает на текстах с грамматическими ошибками, не учитывает наличие знаков препинания. Данная область математики очень обширная и сложная, в будущем, возможно улучшу алгоритм, введя новые математические операции, помогающие ускорить и уточнить поиск.

\pagebreak