\section{Описание}

Алгоритм Форда-Фалкерсона состоит из поиска в ширину и изменение весов графа в соответствии с минимальным весом ребра в найденном пути.

Алгоритм крутит поиск в ширину пока не сможет найти путь из истока в сток графа (1,n). Далее, в двудольном графе меняет веса ребер. Каждый раз находя новый путь в графе, минимльный вес является довабочным потоком в графе и оно прибавляется к общему ответу, когда алгоритм не смодет найти путь, алгоритм прекращает работу и выходит.

Скорость работы алгоритма оценивается как $O(f * |E|)$, где $f$ - кол-во найденных путей в графе, а $|E|$ - кол-во или мощность множества вершин графа. Это можно объяснить так: мы ищем путь в графе, далее для каждой вершины в пути, длина которого может быть максимум $|E|$, меняем веса за $O(1)$.

\pagebreak

\section{Исходный код}

\begin{lstlisting}[language=C++]
// main.cpp
#include <iostream>
#include <vector>
#include "graf.hpp"

using namespace std;

int main() {
    int n, m, from, to, waight;
    cin >> n >> m;
    Graf G(n + 1);

    for (int i = 0; i < m; ++i) {
        cin >> from >> to >> waight;
        G.changeEdge(from, to, waight);
    }

    unsigned long long answer = 0;
    pair <vector <int>, int> path = G.findShortestPath(1, n);

    while (path.second != 0) {
        answer += path.second;
        for (int i = 1; i < (int)path.first.size(); i++) {
            G.changeEdge(path.first[i - 1], path.first[i], G.Waight(path.first[i - 1], path.first[i]) - path.second);
            G.changeEdge(path.first[i], path.first[i - 1], G.Waight(path.first[i], path.first[i - 1]) + path.second);
        }
        path = G.findShortestPath(1, n);
    }
    cout << answer << "\n";
    return 0;
}

\end{lstlisting}


\pagebreak


\begin{lstlisting}[language=C++]
//graf.hpp
#include <iostream>
#include <vector>
#include <unordered_map>
#include <queue>
#include <algorithm>
#include <limits>
using namespace std;

class Graf {
public:
    Graf() = default;
    Graf(int size) : size(size), nodes(size) {}

    void changeEdge(const int from, const int to, const int waight) {
        if (waight == 0) {
            nodes[from].edges.erase(to);
        } else {
            nodes[from].edges[to] = Edge(&nodes[to], waight);
        }
    }

    inline int Waight (const int from, const int to) {
        return nodes[from].edges.find(to) != nodes[from].edges.end() ? nodes[from].edges[to].waight : 0;
    }

    pair <vector <int>, int> /*vector <int> - order of nodes; int - min waight*/ findShortestPath (const int from, const int to) {
        queue <pair <int, Node *>> q; // main queue
        vector <bool> visited(size, false); // vector for visited nodes
        pair <int, Node *> current; // curent value
        vector <int> parent(size, -1); // for each node u have parent node and it index
        pair <vector <int>, int> res({}, 0); // final res data arr of number of noder and min C
        q.push(pair <int, Node *> (from, &nodes[from]));
        visited[from] = true;
        while(!q.empty()) {
            current = q.front(); //make current node in front
            q.pop();    // delete it from queue
            for (pair <int, Edge> edge : (*current.second).edges) {
                // for all next in current node
                // edge.first - number of next node
                // edge.second - edge himself
                if (!visited[edge.first]) {
                    // next node not in visited
                    visited[edge.first] = true;
                    parent[edge.first] = current.first;
                    if (edge.first == to) {
                        // we meet final
                        res.first.push_back(to); // add to res final node
                        res.second = __INT_MAX__; // now min C eq max int (infinity)
                        while (res.first.back() != from) {
                            int curr = res.first.back(), /*current form last*/
                                parent_curr = parent[curr], /*parent of current*/
                                parent_curr_waight = nodes[parent_curr].edges[curr].waight; /*waight of way btn them from parent to relative*/
                            if (parent_curr_waight < res.second) {
                                // if some edge less than min C
                                res.second = parent_curr_waight;
                            }
                            res.first.push_back(parent_curr);
                        }
                        reverse(res.first.begin(), res.first.end());
                        return res;
                    } else {
                        // we not met final
                        q.push(pair <int, Node *> (edge.first, edge.second.node));
                    }
                }
            }
        }
        return res;
    }

    ~Graf() = default;

private:
    struct Node;
    struct Edge {
        Edge() = default;
        Edge(Node * node, const int waight) : node(node), waight(waight) {}
        Node * node;
        int waight;
    };
    struct Node {
        unordered_map <int, Edge> edges;
    };
    int size;
    vector <Node> nodes;
};
\end{lstlisting}

\lstset{language=[gnu] make}
\lstset{
   language=[gnu] make,
   keywordstyle=\color{teal}\textbf,
   stringstyle=\color{blue},
   identifierstyle=\itshape
}

\begin{lstlisting}
CC = g++
CCFLAGS = -std=c++11 -Wall -pedantic -O3
###____###
solution : main.cpp *.hpp ; $(CC) $(CCFLAGS) main.cpp -o solution
clean	 : ;

\end{lstlisting}

\pagebreak

\section{Консоль}

\begin{alltt}
MacBook-Pro-Pavel:src pavelgamov make
g++ -std=c++11 -Wall -pedantic -O3 -g -o main code/main.cpp

valgrind --leak-check=full -v ./main
==59652== Memcheck, a memory error detector
==59652== Copyright (C) 2002-2017, and GNU GPL'd, by Julian Seward et al.
==59652== Using Valgrind-3.14.0-353a3587bb-20181007X and LibVEX; rerun with -h for copyright info
==59652== Command: ./main
==59652== 
--59652-- Valgrind options:
--59652--    --leak-check=full
--59652--    -v
--59652-- Output from sysctl({CTL_KERN,KERN_VERSION}):
--59652--   Darwin Kernel Version 19.2.0: Sat Nov  9 03:47:04 PST 2019; root:xnu-6153.61.1~20/RELEASE_X86_64
5 6
1 2 4
1 3 3
1 4 1
2 5 3
3 5 3
4 5 10
7
==59652== HEAP SUMMARY:
==59652==     in use at exit: 0 bytes in 0 blocks
==59652==   total heap usage: 0 allocs, 0 frees, 0 bytes allocated
==59652== 
==59652== All heap blocks were freed -- no leaks are possible
==59652== 
==59652== ERROR SUMMARY: 0 errors from 0 contexts (suppressed: 1 from 1)
--59652-- 
--59652-- used_suppression:      1 OSX1013:dyld-3 \\ /usr/local/Cellar/valgrind/3.14.0/lib/valgrind/default.supp:1267
==59652== 
==59652== ERROR SUMMARY: 0 errors from 0 contexts (suppressed: 1 from 1)
\end{alltt}

\pagebreak