\section{Описание}

Длинная арифметика — это набор программных средств (структуры данных и алгоритмы), которые позволяют работать с числами гораздо больших величин, чем это позволяют стандартные типы данных.

Основная идея заключается в том, что число хранится в виде массива его цифр.
Цифры могут использоваться из той или иной системы счисления, в данной ЛР применяются десятичная система счисления и её степени: десять тысяч, миллиард.

Операции над числами в этом виде длинной арифметики производятся с помощью "школьных" алгоритмов сложения, вычитания, умножения, деления столбиком. В данной описана работа только с неотрицательными длинными числами.

Также существуют более сложные алгоритмы, которые достигают меньших сложностей, одно из таких - алгоритм быстрого умножения длинных чисел Карацубы, также есть быстрое преобразование фурье, для дальнейшего умножения двух чисел, которые представимы не в виде длинного числа, а набора коифециентов перед членами многочлена, для которого и применяется алгоритм быстрого преобразования Фурье. Для возведения в целую степень был использован алгоритм логарифмического возведения в степень. Он заполняет массив числами от 1-ой степень до $2^{n} \leq d$, где d - нужная нам степень числа. А потом пробегаемся обратно заполняя недостающие значения умножая на уже полученные значения числа в n-ой степени из массива.

\pagebreak

\section{Исходный код}

\begin{lstlisting}[language=C++]
// main.cpp
#include <iostream>
#include <string>
#include <map>
#include "bigint.hpp"

using namespace std;

enum {
    ADD,SUB,MULT,POW,DEL,
    GT,IT,EQ
};

map <string, int> arr { {"+", ADD}, {"-", SUB}, {"*", MULT}, {"^", POW}, {"/", DEL}, {">", GT}, {"<", IT}, {"=", EQ} };

int main () {
    BigInt A, B;
    string oper;
    while (cin >> A >> B >> oper) {

        switch (arr[oper]) {
            case ADD:
                try {
                    cout << A + B << "\n";
                } catch (...) {
                    cout << "Error\n";
                }
                break;
            case SUB:
                try {
                    cout << A - B << "\n";
                } catch (...) {
                    cout << "Error\n";
                }
                break;
            case MULT:
                try {
                    cout << A * B << "\n";
                } catch (...) {
                    cout << "Error\n";
                }
                break;
            case POW:
                try {
                    cout << (A ^ B) << "\n";
                } catch (...) {
                    cout << "Error\n";
                }
                break;
            case DEL:
                try {
                    cout << A / B << "\n";
                } catch (...) {
                    cout << "Error\n";
                }
                break;
            case GT:
                A > B ? cout << "true\n" : cout << "false\n";
                break;
            case IT:
                A < B ? cout << "true\n" : cout << "false\n";
                break;
            case EQ:
                A == B ? cout << "true\n" : cout << "false\n";
                break;
        }

    }
    return 0;
}
\end{lstlisting}

\lstset{language=[gnu] make}
\lstset{
   language=[gnu] make,
   keywordstyle=\color{teal}\textbf,
   stringstyle=\color{blue},
   identifierstyle=\itshape
}

\begin{lstlisting}
CC = g++
CCFLAGS = -std=c++14 -Wall -pedantic -O3
###____###
solution : main.cpp *.hpp ; $(CC) $(CCFLAGS) main.cpp -o solution
clean	 : ;
\end{lstlisting}

\pagebreak

\section{Консоль}

\begin{alltt}
MacBook-Pro-Pavel:src pavelgamov make
g++ -std=c++14 -Wall -pedantic -Werror -Wno-sign-compare -Wno-long-long -o main code/main.cpp

root@pavel:/media/sf_Coding/4sem/Da/6lab/src# valgrind ./main < test/test.test 
==1930== Memcheck, a memory error detector
==1930== Copyright (C) 2002-2017, and GNU GPL'd, by Julian Seward et al.
==1930== Using Valgrind-3.15.0 and LibVEX; rerun with -h for copyright info
==1930== Command: ./main
==1930== 
18720000
==1930== 
==1930== HEAP SUMMARY:
==1930==     in use at exit: 0 bytes in 0 blocks
==1930==   total heap usage: 17 allocs, 17 frees, 82,524 bytes allocated
==1930== 
==1930== All heap blocks were freed -- no leaks are possible
==1930== 
==1930== For lists of detected and suppressed errors, rerun with: -s
==1930== ERROR SUMMARY: 0 errors from 0 contexts (suppressed: 0 from 0)
root@pavel:/media/sf_Coding/4sem/Da/6lab/src# 

\end{alltt}

\pagebreak