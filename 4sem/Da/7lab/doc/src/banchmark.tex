\section{Тест производительности}

Для какого то сравнения используем генетический алгоритм поиска приближенного решения

На тестах до 100 элементов и максимальным значением вместимости рюкзака в 3000, оба алгоритма показывают себя хорошо, генетический алгоритм запрограмирован останавливаться если за 5 итераций не сможет уйти в новый максимум, в таком случае он прекращает работу. 

в выводе моего чекера выводится финальная стоимость предметов, а также их номера, это лишь для проверки генетического алгоритма, так можно понять что генетика не справилась.
Далее пишется время работы, если main - алгоритм через динамическое программирование, sub - через генетику.

Если оба ответа совпали - они не дублируются, а пишется результат и время работы программ.

\begin{alltt}
________
13
1
0.178517103195 main
0.0597698688507 sub
________
33
2
0.0111908912659 main
0.012353181839 sub
________
0

0.0145981311798 main
0.0107469558716 sub
________
1068
1 3 4 5 7 8
0.0137679576874 main
0.0140879154205 sub
________
516
1 2 3 4
0.0117740631104 main
0.0134470462799 sub
\end{alltt}

При тестах на больших примерах - кол-во предметов до 5000, а максимальная вместимость рюказака - до 30000, алгоритм через динамическое программирование показывает худшее время, но лучши результат, часто встречается ситуация, когда генетика не может найти оптимальное решение задачи, из за этого выходит раньше времени, показывая приближенный результат.

\begin{alltt}
________
5670
2 4 6 7 9 11 14 15 16 19 20 21 22 23 26 27 29 33 36 37 38 39 42 44 46 47 48 50 51 52
0.230237960815 main

3473
2 3 5 7 9 11 14 16 19 20 23 24 31 33 37 38 39 42 47 48 49 51 52
0.175779104233 sub
________
413
2 5 7 9 10 13 14
0.00961208343506 main

315
2 3 5 7 8 9 14
0.0130109786987 sub
________
3718
2 6 10 11 13 14 15 16 18 19 24 26 27 29 32 33 34 36 48 63 68 70 76 80 86 92
0.0313041210175 main

540
4 18 20 56 74 76 81 88 90 92
0.0113871097565 sub
________
17334
1 2 3 4 5 6 8 9 10 11 12 13 14 15 16 17 19 20 21 22 23 25 27 28 29 31 32 33 36 37 38 39 40 41 43 44 45 46 48 49 51 52 54 55 57 58 60 61 62 63 64 65 66 67
0.0483999252319 main

11160
3 4 9 10 11 13 14 15 16 20 22 24 25 26 27 28 29 31 32 33 35 38 39 40 41 42 43 45 46 47 48 50 51 52 53 54 56 58 59 60 62 64 65 66 67
0.0116159915924 sub
________
9632
1 2 3 4 5 6 7 8 9 10 11 12 13 14 15 16 17 18 19 20 21 22 23 24 25 26 27 28 29 30 31 32 33 34 35 36 37 38 39 40 41 42 43
0.0266749858856 main
0.0100429058075 sub
\end{alltt}

Что сказать про использование памяти, Динамический алгоритм использует (W * n * n + n) памяти, где первое произведение - наша кубическая матрица, и еще один вектор с весами, чтобы запоминать предметы которые мы обработали.

\pagebreak
