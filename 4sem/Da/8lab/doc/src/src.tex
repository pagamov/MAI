\section{Описание}

Алгоритм запоминает все стороны в массив data, сортирует его по убыванию, а дальше пробегает все троицы значений проверяя на возможность такого треугольника, а также считая площадь. Далее алгоритм имеяя площадь, выводит ее и три ее стороны. Если площадь не найдена, выводит ноль.

Сортировка нужна для быстрой проверки на возможность составить треугольник из трех сторон, но из за этого алгоритм требует O(n) памяти, имея сложность в зависимости от сортировки O(n log n) или O(n).


\pagebreak

\section{Исходный код}

\begin{lstlisting}[language=C++]
#include <iostream>
#include <vector>
#include <algorithm>
#include <cmath>

using namespace std;

bool comp(int& first, int& second) {
    return first > second;
}

bool check_if_valid(const int& s_1, const int& s_2, const int& s_3) {
    if ((s_1 < (s_2 + s_3)) && (s_2 < (s_1 + s_3)) && (s_3 < (s_1 + s_2))) {
        return true;
    } else {
        return false;
    }
}

double area(int s_1, int s_2, int s_3) {
    double p = 0.5 * (s_1 + s_2 + s_3);
    return sqrt(p * (p - s_1) * (p - s_2) * (p - s_3));
}

int main() {
    vector <int> data;
    int n, side, s_1 = 0, s_2 = 0, s_3 = 0;
    double maxArea = 0, curArea;
    cin >> n;
    for (int i = 0; i < n; ++i) {
        cin >> side;
        data.push_back(side);
    }
    sort(data.begin(), data.end(), comp);
    for (int i = 1; i < data.size() - 1; ++i) {
        if (data.size() < 3) {
            break;
        }
        if (check_if_valid(data[i - 1], data[i], data[i + 1])) {
            curArea = area(data[i - 1], data[i], data[i + 1]);
            if (curArea > maxArea) {
                maxArea = curArea;
                s_1 = data[i + 1];
                s_2 = data[i];
                s_3 = data[i - 1];
            }
        }
    }
    if (maxArea == 0) {
        cout << 0 << '\n';
    } else {
        printf("%.3f\n", maxArea);
        cout << s_1 << ' ' << s_2 << ' ' << s_3 << '\n';
    }
    return 0;
}
\end{lstlisting}

\lstset{language=[gnu] make}
\lstset{
   language=[gnu] make,
   keywordstyle=\color{teal}\textbf,
   stringstyle=\color{blue},
   identifierstyle=\itshape
}

\begin{lstlisting}
CC = g++
CCFLAGS = -std=c++11 -Wall -pedantic -O3 -Werror -Wno-sign-compare -Wno-long-long
###____###
solution : main.cpp ; $(CC) $(CCFLAGS) main.cpp -o solution
clean	 : ;
\end{lstlisting}

\pagebreak

\section{Консоль}

\begin{alltt}



root@pavel:/media/sf_Coding/4_sem/Da/8lab/src# make
g++ -std=c++11 -Wall -pedantic -O3 -o main code/main.cpp


root@pavel:/media/sf_Coding/4_sem/Da/8lab/src# valgrind ./main
==1713== Memcheck, a memory error detector
==1713== Copyright (C) 2002-2017, and GNU GPL'd, by Julian Seward et al.
==1713== Using Valgrind-3.15.0 and LibVEX; rerun with -h for copyright info
==1713== Command: ./main
==1713==
4
1
2
3
5
0
==1713==
==1713== HEAP SUMMARY:
==1713== in use at exit: 0 bytes in 0 blocks
==1713== total heap usage: 6 allocs, 6 frees, 74,780 bytes allocated
==1713==
==1713== All heap blocks were freed — no leaks are possible
==1713==
==1713== For lists of detected and suppressed errors, rerun with: -s
==1713== ERROR SUMMARY: 0 errors from 0 contexts (suppressed: 0 from 0)
\end{alltt}

\pagebreak