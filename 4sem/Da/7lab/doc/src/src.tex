\section{Описание}

Данная задача о наполнении рюкзака обычно имеет другое условие, когда общая стоимость считается как $\sum_{i \in I} c_{i}$, а не $|I| * \sum_{i \in I} c_{i}$, из за чего появляется проблема нахождения подобного решения, но для данного условия.

Обычно все решается через таблицу Вес х Номер предмета. Мы берем первый предмет, и смотрим, можем ли положить его в рюкзак весом 0, потом 1, и так до максимального веса рюкзака. Потом берем другой предмет и пытаемся положить его туда же, так как мы знаем какой предмет мы положили до этого, мы можем делать запись текущей стоимости при возможности или невозможности положить предмет в рюкзак.

Пусть A(k,s) есть максимальная стоимость предметов, которые можно уложить в рюкзак вместимости s, если можно использовать только первые k предметов, то есть $ \{ n_{1},n_{2}, \dots ,n_{k} \} $, назовем этот набор допустимых предметов для A(k,s).

A(k,0)=0

A(0,s)=0

Найдем A(k,s). Возможны 2 варианта:

\quad 1. Если предмет k не попал в рюкзак. Тогда A(k,s) равно максимальной стоимости рюкзака с такой же вместимостью и набором допустимых предметов $ \{ n_{1}, n_{2}, \dots , n_{k-1} \} $, то есть A(k,s)=A(k-1,s)

\quad 2. Если k попал в рюкзак. Тогда A(k,s) равно максимальной стоимости рюкзака, где вес s уменьшаем на вес k-ого предмета и набор допустимых предметов $ \{ n_{1},n_{2}, \dots ,n_{k-1} \} $ плюс стоимость k, то есть $A(k-1,s-w_{k})+p_{k}$

\[
A(k,s) = \left\{ \begin{array}{ll}
A(k-1 {,} s) & b_{k}=0\\ 
A(k-1 {,} s - w_{k}) + p_{k} & b_{k}=1\\
\end{array}
\right. \\
\]
  
То есть: $A(k,s)=max(A(k-1,s),A(k-1,s-w_{k})+p_{k})$

Стоимость искомого набора равна A(N,W), так как нужно найти максимальную стоимость рюкзака, где все предметы допустимы и вместимость рюкзака W.

Задача расширяется для трех параметров, мы будем брать новый предмет и для каждого кол-ва и веса проверять на это условие, берем мы предмет или нет. Таким образом наша таблица или матрица превращается в матричный куб.

\pagebreak

\section{Исходный код}

\begin{lstlisting}[language=C++]
// main.cpp
#include <iostream>
#include <vector>
#include <algorithm>

using namespace std;

int main() {
    std::ios::sync_with_stdio(false);
    int n, m, w, c; // n - number of items; m - max mass; w - w_i; c - c_i
    std::cin >> n >> m;
    n++;
    m++;
    vector<vector<vector<int>>> matrix(n, vector<vector<int>>(n, vector<int>(m, 0)));
    vector<int> shift(n); // for items
    for (int k = 1; k < n; k++) { // layer of item from 1 to n
        cin >> w >> c; //get w and c of item
        shift[k] = w; //add item
        for (int j = 1; j <= k; j++) { // for every variant of number to take
            for (int c_m = 0; c_m < m; c_m++) { // for every mass in range
                if (c_m - w >= 0) { // we can insert item
                    if (matrix[k - 1][j - 1][c_m - w] != 0 || j - 1 == 0) {
                        matrix[k][j][c_m] = max(matrix[k - 1][j][c_m], matrix[k - 1][j - 1][c_m - w] + c);
                    } else {
                        matrix[k][j][c_m] = matrix[k - 1][j][c_m];
                    }
                } else { // we cannot insert item
                    matrix[k][j][c_m] = matrix[k - 1][j][c_m];
                }
            }
        }
    }

    int k = n - 1, c_c = 0, c_m = m - 1; // k - item level; c_c = curent count; c_m - curent mass
    int max_value = 0; // max item value based on count
    for (int count = 0; count < n; count++) {
        if (matrix[k][count][c_m] * count > max_value) {
            max_value = matrix[k][count][c_m] * count;
            c_c = count;
        }
    }
    cout << max_value << "\n";

    vector<int> res;
    while (c_c > 0) {
        if (matrix[k][c_c][c_m] != matrix[k - 1][c_c][c_m]) {
            c_c--;
            c_m -= shift[k];
            res.push_back(k);
        }
        k--;
    }

    for (int r = res.size() - 1; r >= 0; r--) {
        cout << res[r] << ' ';
    }
    cout << '\n';
    return 0;
}
\end{lstlisting}

\lstset{language=[gnu] make}
\lstset{
   language=[gnu] make,
   keywordstyle=\color{teal}\textbf,
   stringstyle=\color{blue},
   identifierstyle=\itshape
}

\begin{lstlisting}
CCFLAGS = -std=c++11 -Wall -pedantic -O3 -Werror -Wno-sign-compare -Wno-long-long
###____###
solution : main.cpp ; $(CC) $(CCFLAGS) main.cpp -o solution
clean	 : ;
###___###
\end{lstlisting}

\pagebreak

\section{Консоль}

\begin{alltt}
pagamov@pagamov-VirtualBox:~/Desktop/Da/7lab/src valgrind ./main 
==3407== Memcheck, a memory error detector
==3407== Copyright (C) 2002-2017, and GNU GPL'd, by Julian Seward et al.
==3407== Using Valgrind-3.13.0 and LibVEX; rerun with -h for copyright info
==3407== Command: ./main
==3407== 
5 20
4 5
2 4
6 1
2 3
5 1
70
1 2 3 4 5 
==3407== 
==3407== HEAP SUMMARY:
==3407==   total heap usage: 63 allocs, 63 frees, 200,432 bytes allocated
==3407== 
==3407== LEAK SUMMARY:
==3407==    definitely lost: 0 bytes in 0 blocks
==3407==    indirectly lost: 0 bytes in 0 blocks
==3407==      possibly lost: 0 bytes in 0 blocks
==3407==    still reachable: 0 bytes in 0 blocks
==3407==         suppressed: 0 bytes in 0 blocks
==3407== Rerun with --leak-check=full to see details of leaked memory
==3407== 
==3407== For counts of detected and suppressed errors, rerun with: -v
==3407== ERROR SUMMARY: 0 errors from 0 contexts (suppressed: 0 from 0)
pagamov@pagamov-VirtualBox:~/Desktop/Da/7lab/src
\end{alltt}

\pagebreak