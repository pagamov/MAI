\section{Выводы}

Структура B-дерева применяется для организации индексов во многих современных СУБД.

B-дерево может применяться для структурирования (индексирования) информации на жёстком диске (как правило, метаданных). Время доступа к произвольному блоку на жёстком диске очень велико (порядка миллисекунд), поскольку оно определяется скоростью вращения диска и перемещения головок. Поэтому важно уменьшить количество узлов, просматриваемых при каждой операции. Использование поиска по списку каждый раз для нахождения случайного блока могло бы привести к чрезмерному количеству обращений к диску вследствие необходимости последовательного прохода по всем его элементам, предшествующим заданному, тогда как поиск в B-дереве, благодаря свойствам сбалансированности и высокой ветвистости, позволяет значительно сократить количество таких операций.

Относительно простая реализация алгоритмов и существование готовых библиотек (в том числе для C) для работы со структурой B-дерева обеспечивают популярность применения такой организации памяти в самых разнообразных программах, работающих с большими объёмами данных.

Во всех случаях полезное использование пространства вторичной памяти составляет свыше 50 \% . С ростом степени полезного использования памяти не происходит снижения качества обслуживания.

Произвольный доступ к записи реализуется посредством малого количества подопераций (обращения к физическим блокам).

В среднем достаточно эффективно реализуются операции включения и удаления записей; при этом сохраняется естественный порядок ключей с целью последовательной обработки, а также соответствующий баланс дерева для обеспечения быстрой произвольной выборки.

Неизменная упорядоченность по ключу обеспечивает возможность эффективной пакетной обработки.

Основной недостаток В-деревьев состоит в отсутствии для них эффективных средств выборки данных (т.е. метода обхода дерева), упорядоченных по отличному от выбранного ключа.

\pagebreak