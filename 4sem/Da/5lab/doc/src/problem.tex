\CWHeader{Лабораторная работа \textnumero 4}

\CWProblem{

Вариант №5
Необходимо реализовать алгоритм Укконена построения суффиксного дерева за линейное время. Построив такое дерево для некоторых из выходных строк, необходимо воспользоваться полученным суффисным деревом для решения своего варианта задания.

Алфавит строк: строчные буквы латинского алфавита (т.е. от a до z).

Вариант:

Найти самую длинную общую подстроку двух строк.

Формат входных данных: 

Две строки.

Формат результата:

На первой строке нужно распечатать длину максимальной общей подстроки, затем перечислить все возможные варианты общих подстрок этой длины в порядке лексикографического возрастания без повторов.

Примеры:

Входные данные:

xabay

xabcbay

Результат работы:

3

bay

xab
}

\pagebreak